%% UTF-8
\documentclass[twoside,UTF8,phd]{nputhesis}

% \RequirePackage[%pdftex,
%             CJKbookmarks=true,
%             pdfstartview=FitH,
%             bookmarksnumbered=true,
%             bookmarksopen=true,
%             colorlinks=true,
%             citecolor=black,
%             linkcolor=black,
%             anchorcolor=black,
%             urlcolor=black
%             ]{hyperref}
\usepackage{lipsum}


\newtheorem{thm}{定理}[section]
\theoremstyle{plain}
\newtheorem{oldthm}{旧定理}[section]

\schoolno{10699}
\classno{O.242}
\secretlevel{公开}
\authorno{2018999999}

\title[\LaTeX\ Template for Thesis of NPU]{西北工业大学硕博士论文\LaTeX 模板}

\author[San Zhang]{张\,\,三}
\major[Mathematics]{数学}
\supervisor[Si Li]{李四}
\applydate[April 2018]{2018~年~4~月}
\support{本文研究得到某某基金(编号:XXXXXXX)资助。}

\begin{document}
\makecover  % 中英文封面 
\frontmatter

% 中文摘要
\begin{abstract}  
    因为\TeX 具有出色的公式和图表排版功能, 所以大部分期刊都要求作者投稿时使用
    \TeX. 学生写论文时也多用 \TeX, 所以我们制作本模板以节省写毕业论文的时间 
    (重用之前编辑的公式和图标).

    本文简要介绍西北工业大学论文模板 (nputhesis) 的实现和使用.

    { % 乱码测试
        \noindent\hrulefill\\
        {\centerline {\it 乱码模式开启}}
        \lipsum[1-5]
        {\centerline{\it 乱码模式关闭}}
        \noindent\hrulefill
    }
    \begin{keywords}
        论文模板, \LaTeX, 西工大 
    \end{keywords}
\end{abstract}

% 英文摘要
\begin{Abstract}
    { % some meaningless words.
        \noindent\hrulefill\\
        {\centerline {\it 乱码模式开启}}
        \lipsum[1-4]
        {\centerline{\it 乱码模式关闭}}
        \noindent\hrulefill
    }
    \begin{Keywords}
        Thesis Template, \LaTeX, NPU
    \end{Keywords}
\end{Abstract}

% 目录
\tableofcontents 

\mainmatter  % 
\chapter{nputhesis 简介}

\section{\TeX 和 \LaTeX 介绍}
关于 \TeX 和 \LaTeX 请参考 \cite{Knuth1986,Lamport1994,Liu2013}, 其中 \cite{Liu2013} 最适合入门.  

\section{nputhesis 依赖}
如下表格给出了测试编译通过的环境
\begin{table}[h]
  \caption{测试环境\cite{Liu2013}}
    \centering
    \begin{tabular}{cccc}
        \toprule
        操作系统    & \TeX 系统   & 版本  & 引擎\\
        \midrule
        Windows 10  & TeXLive     & 2017  & xelatex\\
        \bottomrule
    \end{tabular}
\end{table}
\lipsum[9-15]
\section{插入图片}
\lipsum[1-3]
\begin{figure}
    \centering
    \includegraphics[width=0.3\textwidth]{figures/fig1-1.pdf}
    \includegraphics[width=0.3\textwidth]{figures/fig1-1.pdf}
    \caption{样例图}
\end{figure}
\begin{figure}
    \centering
    \includegraphics[width=0.5\textwidth]{figures/fig2.pdf}
    \caption{另一个图}
\end{figure}
\lipsum[4-6]
\section{定理环境}
\lipsum[1]
\begin{thm}
    这里是定理环境使用 `nputheorem' 格式.
\end{thm}
\lipsum[2]
\begin{oldthm}
    这里是 `amsthm' 默认的定理环境, 使用 `plain' 格式.
\end{oldthm}
\lipsum[3]

\chapter{实现}
\section{思路}
\section{代码}


\backmatter
\bibliographystyle{nputhesis}
\bibliography{ref}

\Appendix
This is appendix.

\Thanks
This is a thanks.

\Work
% TODO 如何直接引用使参考文献的内容显示在这里

\statement
\end{document}
