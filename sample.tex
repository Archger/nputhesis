%% UTF-8
\documentclass[twoside,UTF8,phd]{nputhesis}

\usepackage{booktabs}
\heavyrulewidth=2.25bp
\lightrulewidth=1bp
\usepackage{tabularx}       

\usepackage{multirow}
\usepackage{graphicx}
\usepackage{lipsum}
\newtheorem{thm}{定理}[section]
\theoremstyle{plain}
\newtheorem{oldthm}{旧定理}[section]

\schoolno{10699}
%\classno{}
%\secretlevel{}
\title[\LaTeX\ Template for Thesis of NPU]{西北工业大学硕博士论文\LaTeX 模板}
\author[San Zhang]{张\,\,三}
\authorno{2014000000}
\major[Applied Mathematics]{应用数学}
\supervisor[Si Li]{李四}
\applydate[September 2017]{2017~年~3~月}
\support{本文研究得到某某基金(编号:XXXXXXX)资助。}

\begin{document}
\makecover  % 中英文封面 
\frontmatter

% 中文摘要
\begin{abstract}  
    因为\TeX 具有出色的公式和图表排版功能, 所以大部分期刊都要求作者投稿时使用
    \TeX. 学生写论文时也多用 \TeX, 所以我们制作本模板以节省写毕业论文的时间 
    (重用之前编辑的公式和图标).

    本文简要介绍西北工业大学论文模板 (nputhesis) 的实现和使用.

    { % 乱码测试
        \noindent\hrulefill\\
        {\centerline {\it 乱码模式开启}}
        \lipsum[1-5]
        {\centerline{\it 乱码模式关闭}}
        \noindent\hrulefill
    }
    \begin{keywords}
        论文模板, \LaTeX, 西工大 
    \end{keywords}
\end{abstract}

% 英文摘要
\begin{Abstract}

    { % some meaningless words.
        \noindent\hrulefill\\
        {\centerline {\it 乱码模式开启}}
        \lipsum[1-4]
        {\centerline{\it 乱码模式关闭}}
        \noindent\hrulefill
    }
    \begin{Keywords}
        Thesis Template, \LaTeX, NPU
    \end{Keywords}
\end{Abstract}

% 目录
\tableofcontents 
\mainmatter  % 
\setlength{\baselineskip}{20pt}

\chapter{nputhesis 简介}

\section{\TeX 和 \LaTeX 介绍}
关于 \TeX 和 \LaTeX 请参考 \cite{Knuth1986,Lamport1994,Liu2013}, 其中 \cite{Liu2013} 最适合入门.  

\section{nputhesis 依赖}
如下表格给出了测试编译通过的环境
\begin{table}[h]
  \caption{测试环境\cite{Liu2013}}
    \centering
    \zihao{5}
    \begin{tabularx}{\linewidth}{*{4}{>{\centering\arraybackslash}X}}
        \toprule
        {\bf 操作系统}    & \TeX 系统   & 版本  & 引擎\\
        \midrule[1bp]
        Windows 10  & TeXLive     & 2017  & xelatex\\
        \bottomrule
    \end{tabularx}
\end{table}
\section{表格环境}
\begin{table}[h]
	\caption{聚合物基复合材料性能}
	\centering
	\zihao{5}
	\begin{tabularx}{0.98\linewidth}{*{5}{>{\centering\arraybackslash}X}}
		\toprule
		\multirow{2}*{材料}   & \multicolumn{2}{c}{碳/环氧}    & \multicolumn{2}{c}{玻璃/环氧}\\
		\cmidrule{2-3} \cmidrule{4-5}
		&纵向	&横向	&纵向	&横向\\
		\midrule
模量,GPa&	181	&10.3&	38.6&	8.3
 \\
压缩强度,MPa&	1500&	246&	610	&118
\\
拉伸强度,MPa&	1500&	40&	1062&	31
\\
		\bottomrule
	\end{tabularx}
\end{table}

\lipsum[9-15]
\section{插入图片}
\includegraphics[width=6.67cm, height=5cm]{figures/fig1-1.pdf}\\
\includegraphics[width=9cm, height=6.75cm]{figures/fig1-2.pdf}\\
\includegraphics[width=13.5cm, height=9cm]{figures/fig2.pdf} \par
\includegraphics[width=6.67cm]{figures/fig1-1.pdf}\\
\includegraphics[width=9cm]{figures/fig1-2.pdf}\\
\includegraphics[width=13.5cm]{figures/fig2.pdf}
\section{定理环境}
\lipsum[1]
\begin{thm}
    这里是定理环境使用 `nputheorem' 格式.
\end{thm}
\lipsum[2]
\begin{oldthm}
    这里是 `amsthm' 默认的定理环境, 使用 `plain' 格式.
\end{oldthm}
\lipsum[3]

\chapter{实现}
\section{思路}
\section{代码}
\begin{equation}
    u(x) = \int_{\Gamma} f(x)dx
\end{equation}

\backmatter
\bibliographystyle{nputhesis}
\bibliography{ref}

\Appendix
This is appendix.

\Thanks
This is a thanks.

\Work
% TODO 如何直接引用使参考文献的内容显示在这里

\statement
\end{document}
